\chapter{Conclusions}
The goal of this thesis was to design a set of controllers for the underactuated fast and extremely non-linear process, which was Furuta's pendulum. For that purpose Swing-Up, LQR, MPC and NMPC controllers were designed.\\
In the beginning, we got the nonlinear dynamic model of the process. But only NMPC controller can control the process, which is described by such model. To make it suitable for Swing-Up, LQR and MPC controllers we had to linearise it relative to two operating points: at the downside position, where the process is stable, and at the upright position, where the process is unstable. The reason for that is that the process is controlled by LQR and MPC controllers when the pendulum is in the upper position, and it is controlled by a Swing-Up controller when the pendulum is in the lower position.\\
All controllers were designed using MATLAB and also tested in simulations. By evaluating simulation results we can conclude that all controllers were designed properly, and could be used to control the real process. 
