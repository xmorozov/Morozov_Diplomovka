\chapter{Introduction}
\label{ch:intro}
Nowadays Model Predictive Control (MPC) is probably the most advanced control strategy available on the market. This method is based on optimization, which is its first big advantage. Another, probably even bigger advantage of MPC, is that different physical limitations of the process and constrains on measured variables, control variables, and controlled variables are the part of that optimization problem. And if that optimization problem is solvable, we can guarantee, that those constraints would not be violated during the plant operation. As it is a predictive control strategy, it uses accurate model predictions of the future behavior of the system and can provide early warnings of potential problems.

As we can see an MPC is a great control strategy. But nothing in this world is for free and MPC is not an exception. MPC is expensive, both in time and resources. MPC solves an optimization problem. Often with equality and inequality constraints. Sometimes those constraints could be even nonlinear, and such optimization problems might be extremely difficult to solve. So pretty strong computation machines are required. MPC also requires an appropriate process model to compute accurate predictions. And the main industrial restriction on MPC is time. Most of the industrial problems are pretty big and complex, and MPC must be able to solve them during only one sampling period.

This thesis is dedicated to designing a predictive controller, which deals with the main restrictions on MPC such as short available time to solve an optimization problem and nonlinear constraints. To test such MPC the Furuta Pendulum is used. Furuta pendulum is a rotational inverted pendulum and it is a great example of a fast, nonlinear process with complex dynamics. It was invented in 1992 at the Tokyo Institute of Technology by Katsuhisa as an example of a complex nonlinear oscillator, on the purpose of process control algorithm testing. This pendulum is underactuated and extremely non-linear due to gravitational forces and Coriolis interactions. As the main task of the thesis is to perform a swing-up control of the pendulum, two different approaches are used: first is a heuristic swing-up control, where the pendulum is controlled by two controllers: energy-shaping controller and a linear MPC. The second approach is an optimal swing-up control, where a Nonlinear Model Predictive Control (NMPC) strategy is used.

The thesis consists of two main parts: the first is a theoretical foundation, and the second illustrates simulation control of the pendulum device. In the first part operation of the pendulum is explained and its mathematical model derived. Also in that part, the theory for individual controllers is described. In the second part, both heuristic and optimal swing-up control approaches are explained as well as \textit{MATLAB} implementation, and the results of the control are illustrated.   
