\chapter{Conclusions}
In this thesis, we have developed a control strategies to control a complex nonlinear oscillator what the Furuta pendulum is. The main task was to perform a Swing-Up control of the pendulum, with its stabilization at the upright position in an optimal way. But before we could proceed to the control strategies and control attempts, the mathematical dynamic model of the pendulum must be obtained first. For that purpose, a Lagrangian formulation of the system dynamics of the mechanical system was used. The Lagrangian formulation requires the construction of Lagrangian and Euler-Lagrange equations. As the Furuta pendulum has two degrees of freedom, two Euler-Lagrange equations were constructed. Then, by solving those equations, two equations of motion for both the Arm And the Pendulum were obtained. Further from those equations of motion, the nonlinear continuous-time State-Space dynamic model of the pendulum was derived. Then by linearizing that nonlinear model at the upright operation point, the linear dynamic model was acquired.

The next part of the thesis is a theoretical foundation for the individual controllers, used to control the pendulum. There are in total three controllers a Model Predictive Controller, an Energy-Shaping controller, and a Nonlinear Model Predictive Controller. For MPC was shown how to transform a problem of state regulation to the origin, to the standard Quadratic Programming problem. For Energy-Shaping controller was discussed the control law, which allows implementing such a state-feedback controller. For NMPC was shown how to solve a Nonlinear Programming problem by a numerical method, called Sequential Quadratic Programming.

The main task of the thesis is to perform Heuristic Swing-Up control and Optimal Swing-Up control of the pendulum. For Heuristic Swing-Up control an MPC and Energy-Shaping controllers were used. Initially, a Swing-Up controller is used to bring the pendulum in a range around the upright operating point, where the linear dynamic model is valid. Then the active controller switches to MPC, which further stabilizes the pendulum at the upright position. For the Optimal Swing-Up control the NMPC strategy was used. NMPC was designed through the \textsc{Matmpc} toolbox.

All controllers were designed and tested in \textsc{Matlab}. The obtained results are closer discussed in individual sections of the respective chapter.