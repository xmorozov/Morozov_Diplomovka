\chapter{Conclusions}
In this thesis, we have developed strategies to control a complex nonlinear oscillator that the Furuta pendulum is. The main task was to perform a Swing-Up control of the pendulum, with its stabilization at the upright position in an optimal way. But before we could proceed to the control strategies and control attempts, the mathematical dynamic model of the pendulum had to be obtained first. For that purpose, a Lagrangian formulation of the system dynamics of the mechanical system was used. The Lagrangian formulation required the construction of Lagrangian and Euler-Lagrange equations. As the Furuta pendulum has two degrees of freedom, two Euler-Lagrange equations were constructed. Then, by solving those equations, two equations of motion for both the arm and the pendulum were obtained. Further from those equations of motion, the nonlinear continuous-time State-Space dynamic model of the pendulum was derived. Then by linearizing that nonlinear model at the upright operation point, the linear dynamic model was acquired.

The next part of the thesis is a theoretical foundation for the individual controllers, used to control the pendulum. There are in total three controllers. A Model Predictive Controller, an Energy-Shaping controller, and a Nonlinear Model Predictive Controller. For MPC it was shown how to transform a problem of state regulation to the origin, to the standard Quadratic Programming problem. For Energy-Shaping controller the control law was discussed, which allowed implementing of such a state-feedback controller. For NMPC it was shown how to solve a Nonlinear Programming problem by a numerical method, called Sequential Quadratic Programming.

The main task of the thesis was to perform Heuristic Swing-Up control and Optimal Swing-Up control of the pendulum. For Heuristic Swing-Up control an MPC and Energy-Shaping controllers were used. Initially, a Swing-Up controller was used to bring the pendulum in the range around the upright operating point, where the linear dynamic model was valid. Then the active controller switched to MPC, which further stabilized the pendulum in the upright position. For the Optimal Swing-Up control the NMPC strategy was used. NMPC was designed through the \textsc{Matmpc} toolbox.

All controllers were designed and tested in \textsc{Matlab}. The obtained results are closer discussed in individual sections of the respective chapter.