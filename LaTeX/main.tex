\documentclass[english]{iamthesis}
% changle "slovak" to "english" for english version of the thesis

%----------------------------------------------------------------%
% THESIS DATA

% student name with title e.g. Ing. Martin Klaučo
\def\thesisauthor{Alexey Morozov} 

% year of submmiting to AIS
\def\thesisyear{2020}

% registration number generated by AIS e.g. 19990-50920
\def\thesisnumber{NaN} 

% thesis type: BACHELOR|MASTER|DISSERTATION or in slovak 
% BAKALÁRSKA|DIPLOMOVÁ|DIZERTAČNÁ
\def\thesistype{SEMESTRAL}

% thesis title
\def\thesistitle{Nonlinear Model Predictive Control of Rotary Pendulum}

% thesis supervisor including degrees e.g. Ing. Martin Klaučo, PhD.
\def\thesissupervisor{Ing. Martin Klaučo, PhD.}

% study field (translate to english if neccesarry) e.g. "Riadenie Procesov" or
% "Process Control"
\def\thesisprogram{Process Control}

% Institute (translate to english if neccesary)
% e.g., "Institute of Information Engineering, Automation, and Mathematics"
\def\thesisinst{Institute of Information Engineering, Automation, and Mathematics}

% Title of the Acknowledgment
% For slovak write: "Poďakovanie" for English write: "Acknowledgment"
\def\thesisack{Acknowledgment}


% End THESIS DATA
%----------------------------------------------------------------%

%----------------------------------------------------------------%
%   Titles and other stuff                                       %
%----------------------------------------------------------------%
\author{\thesisauthor}
\title{\thesistitle}
\date{\today}
\usepackage{layouts}
\usepackage{layout}
%----------------------------------------------------------------%
%   Let the document begin                                       %
%----------------------------------------------------------------%
\begin{document}

% ---------------------------------------------------------------%
% The Frontmatter  !! Do NOT change the structure !!             % 
%----------------------------------------------------------------%

\coverpage

\frontmatter
\pagenumbering{roman}

% include assignment generated by AIS system
%\includepdf[page=1]{content/assignment.pdf}
% use this command only if your assignment has more than 2 pages
% \includepdf[page=2]{content/assignment.pdf}


% do not remove following commands
% do not following commands
%\chapter*{\thesisack}
%\markboth{}{}
%\addcontentsline{toc}{chapter}{\thesisack}
%Martin, Matus, to be continued...

\chapter*{Abstract}
\markboth{}{}
\addcontentsline{toc}{chapter}{Abstract}
This work is dedicated to designing control strategies for the rotational inverted pendulum or Furuta’s pendulum. Those strategies are Linear-quadratic regulator (LQR) with the Swing-up controller, Model Predictive Control (MPC) method with the Swing-up controller and the Nonlinear Model Predictive Control (NMPC) strategy. Those controllers are designed in a computing environment MATLAB, for NMPC has used MATMPC toolbox as well as the CasADi toolbox. After verifying in simulations those controllers are used to control the real process.  


%\chapter*{Abstrakt}
%\markboth{}{}
%\addcontentsline{toc}{chapter}{Abstrakt}
%Tato diplomová práca je venovaná návrhu nelinearneho, pediktivneho regulátora pre riadenie inverzného, rotačného kyvadla, taktiež známeho ako Furutovho kyvadlo. Hlavnou úlohou je vývoj riadiaceho algoritmu, ktorý je schopný zabespecit Swing-Up kyvadla s jeho nasledujúcou stabilizáciou v hornej polohe. V tejto práce navrhujeme dva riagiace algoritmy heuristicky Swing-Up a optimálny Swing-Up. Obidva algorimy vyžadujú znalosť matematického modelu kyvadla, ktorí odvádzame pomocou Euler-Lagrangeovych rovníc. Následne navrhujeme Energy-Shaping regulátor a lineárny prediktivny regulátor, ktoré využíva heuristicky Swing-Up algoritm a nelinearny prediktivny regulátor pre optimálny Swing-Up. Druha časť práce je venovaná prípadovéj štúdie, kde pomocou týchto stratégií riadime matematickú reprezentáciu Furutovho kyvadla. 

\setcounter{tocdepth}{2}
\renewcommand{\baselinestretch}{0.1}\normalsize
\tableofcontents
\renewcommand{\baselinestretch}{1.1}\normalsize

% ----------------------------------------------------------------%
% The Mainmatter !! Do NOT change the structure!!                 %
% ----------------------------------------------------------------%
\mainmatter

% individual chapters should be included via a separate tex file, as shown in 
% here. When working in TexStudio (recomennded tool for Win and Mac) set the 
% main.tex as an Explicit root document, so you can compile even you are
% working on other chapter in other tex file.
%
% Open main.tex THEN click Options-> Root Document -> Set Current Document as
% Explicit Root

% introduction
\chapter{Introduction}
\label{ch:intro}
Nowadays Model Predictive Control (MPC) perhaps the most advanced control strategy available on the market. This method is based on optimization, thus mathematically provable, that  the calculated control inputs ensure the best control performance for the present scenario, which is a big advantage. Another, perhaps an even bigger advantage of MPC, is that different physical limitations of the process and constrains on measured variables, control variables, and controlled variables are a part of that optimization problem. And if the optimization problem is feasible, we can guarantee, that those constraints would not be violated during the plant operation. As it is a predictive control strategy, it uses accurate model predictions of the future behavior of the system and can provide early warnings of potential problems.

As we can see an MPC is a great control strategy. But nothing in this world is for free and MPC is not an exception. The MPC strategy is expensive, both in time and resources. MPC solves an optimization problem. This optimization problem could also be both equality and inequality constrained. And the presence of these constraints can make the optimization problem extremely difficult to solve. Therefore pretty strong computation machines are required. The MPC also requires an appropriate process model to compute accurate predictions. And the main industrial restriction on MPC is time. Most of the industrial problems are really complicated because of the high number of process and manipulated variables and their intercations. And the MPC must be able to solve such problems during only one sampling period.

This thesis is dedicated to designing a predictive controller, that deals with the main restrictions on MPC such as short available time to solve an optimization problem and nonlinear constraints. To test such MPC the Furuta Pendulum is used. Furuta pendulum is a rotational inverted pendulum and it is a great example of a fast, nonlinear process with complex dynamics. It was invented in 1992 at the Tokyo Institute of Technology by Katsuhisa Furuta as an example of a complex nonlinear oscillator, with the purpose of process control algorithm testing. This pendulum is underactuated and extremely non-linear due to gravitational forces and Coriolis interactions. As the main task of the thesis is to perform a swing-up control of the pendulum, two different approaches are used: first is a heuristic swing-up control, where the pendulum is controlled by two controllers: energy-shaping controller and a linear MPC. The second approach is an optimal swing-up control, where a Nonlinear Model Predictive Control (NMPC) strategy is used.

The thesis consists of two main parts: the first is a theoretical foundation, and the second illustrates simulation control of the pendulum device. In the first part operation of the pendulum is explained and its mathematical model derived. Also in that part, the theory for individual controllers is described. In the second part, both heuristic and optimal swing-up control approaches are explained as well as \textsc{MATLAB} implementation, and the results of the control are illustrated.   


% theory
\chapter{Theory}
\section{Furuta's Pendulum}
Rotational inverted pendulum or Furuta’s pendulum was firstly invented in 1992 at the Tokyo Institute of Technology by Katsuhisa Furuta as an example of a complex nonlinear oscillator in the sake of control algorithms testing in process control theory.
The pendulum composes of two main parts: motor-driven arm, which rotates in the horizontal plane and a pendulum, attached to that arm, which freely rotates in the vertical plane. The system is underactuated and extremely nonlinear due to the gravitational forces and the coupling arising from the Coriolis and centripetal forces. The schematic representation of the pendulum is shown in \ref{furuta}
\begin{figure}[h]
	\centering
	\includegraphics[width=.6\linewidth]{images/furuta}
	\caption{Furuta's Pendulum}
	\label{furuta}
\end{figure}
\newpage
The symbols in the figure indicate the following:
\begin{itemize}
	\item \textbf{$g$} - gravitational acceleration [\si{\metre\per\square\second}]
	\item \textbf{$m_0$} - mass of arm [\si{\kilogram}]
	\item \textbf{$m_1$} - mass of pendulum [\si{\kilogram}]
	\item \textbf{$L_0$} - length of arm [\si{\metre}]
	\item \textbf{$L_1$} - length of pendulum [\si{\metre}]
	\item \textbf{$l_1$} - location of the pendulums center of mass [\si{\metre}]
	\item \textbf{$I_0$} - moment of inertia of arm [\si{\kilogram\per\square\metre}]
	\item \textbf{$I_1$} - moment of inertia of pendulum [\si{\kilogram\per\square\metre}]
	\item \textbf{$\theta_0$} - arm angle [\si{\radian}]
	\item \textbf{$\theta_1$} - pendulum angle [\si{\radian}]
	\item \textbf{$\tau$} - motor torque [\si{\volt}]
\end{itemize}
\subsection{Mathematical Model of the Pendulum}
To design a predictive controller, the knowledge of the dynamic model of the process is necessary. For that purpose, we found the discrete-time state-space representation the most suitable.
To obtain the state representation of the process the state variables must be defined first:
\begin{equation}
	\begin{bmatrix}
	x_1 & x_2 & x_3 & x_4
	\end{bmatrix}^\intercal = 
	\begin{bmatrix}
	\theta_0 & \dot{\theta_0} & \theta_1 & \dot{\theta_1}
	\end{bmatrix}^\intercal
\end{equation}
And the control variable:
\begin{equation} u = \tau \end{equation}
The analytical model is based on the equations of motion derived from the Lagrange equations of the second kind, which represent the most commonly used method for establishing equations of motion of difficult mechanical systems. This method is applied for obtaining mathematical models of eg manipulators, with several degreases of freedom. Taking into account damping in the
form of friction, the resulting equations of motion of the
pendulum are 
\begin{subequations}
	\begin{align}
	\ddot{\theta}_0 &= \frac{\gamma(\epsilon\dot{\theta_0}^2+\rho)-\delta(\tau+\beta\dot{\theta_1}^2-\sigma\dot{\theta_0}\dot{\theta_1})}{\gamma^2-\alpha\delta}\\
	\ddot{\theta}_1 &= \frac{\gamma(\tau+\beta\dot{\theta_1}^2-\sigma\dot{\theta_0}\dot{\theta_1})-\alpha(\epsilon\dot{\theta_0}^2+\rho)}{\gamma^2-\alpha\delta}
	\end{align}
\end{subequations}
where
\begin{subequations}
	\begin{align}
	\alpha &= I_0+L_0^2m_1+l_1^2m_1\sin^2\theta_1\\
	\beta &= L_0m_1l_1\sin\theta_1 \\
	\gamma &= L_0m_1l_1\cos\theta_1\\
	\delta &= I_1+l_1^2m_1\\
	\epsilon &= l^2_1m_1\sin\theta_1\cos\theta_1\\
	\rho &= m_1gl_1\sin\theta_1\\
	\tau &= 2l^2_1m_1\sin\theta_1\cos\theta_1
	\end{align}
\end{subequations}
These equations are further utilized to establish the state
space model
\begin{subequations}
	\begin{align}
\dot{x_1} &= \dot{\theta_0} \\
\dot{x_2} &= \frac{\gamma(\epsilon\dot{\theta_0}^2+\rho)-\delta(\tau+\beta\dot{\theta_1}^2-\sigma\dot{\theta_0}\dot{\theta_1})}{\gamma^2-\alpha\delta}\\
\dot{x_3} &= \dot{\theta_1}\\
\dot{x_4} &= \frac{\gamma(\tau+\beta\dot{\theta_1}^2-\sigma\dot{\theta_0}\dot{\theta_1})-\alpha(\epsilon\dot{\theta_0}^2+\rho)}{\gamma^2-\alpha\delta}
	\end{align}
\end{subequations}

Now these non-linear differential equations we can write in the form of matrices:
\begin{equation}\label{nonlinmodel}
\begin{bmatrix}
\dot{x_1} \\ \dot{x_2} \\ \dot{x_3} \\ \dot{x_4}
\end{bmatrix} = \begin{bmatrix}
\dot{\theta_0}\\
\frac{\gamma(\epsilon\dot{\theta_0}^2+\rho)-\delta(\tau+\beta\dot{\theta_1}^2-\sigma\dot{\theta_0}\dot{\theta_1})}{\gamma^2-\alpha\delta}\\
\dot{\theta_1}\\
 \frac{\gamma(\tau+\beta\dot{\theta_1}^2-\sigma\dot{\theta_0}\dot{\theta_1})-\alpha(\epsilon\dot{\theta_0}^2+\rho)}{\gamma^2-\alpha\delta}
\end{bmatrix}
\end{equation}
As we can see the derivatives of the states are the functions of the current states and control input. And more importantly, those variables have nonlinear interactions within each dynamic equation. 
\begin{equation}\begin{bmatrix}
\dot{x_1} \\ \dot{x_2} \\ \dot{x_3} \\ \dot{x_4}
\end{bmatrix} = f(x,u) =\begin{bmatrix}f_1(x,u)\\f_2(x,u)\\f_3(x,u)\\f_4(x,u)\end{bmatrix} \end{equation}
So that’s our non-linear dynamic model of the process. But only the NMPC controller is able to operate with such a model.  So, to make that model suitable for LQR and MPC controller we can approximate that non-linear model by a linear model as follows:
\begin{equation}\dot{x} = Ax + Bu\end{equation}
And the constant matrices are derived as:
\begin{equation}
A = \begin{bmatrix}
\frac{\partial f_1(x,u)}{\partial x_1}&\frac{\partial f_1(x,u)}{\partial x_2}&\frac{\partial f_1(x,u)}{\partial x_3}&\frac{\partial f_1(x,u)}{\partial x_4}\\
\frac{\partial f_2(x,u)}{\partial x_1}&\frac{\partial f_2(x,u)}{\partial x_2}&\frac{\partial f_2(x,u)}{\partial x_3}&\frac{\partial f_2(x,u)}{\partial x_4}\\
\frac{\partial f_3(x,u)}{\partial x_1}&\frac{\partial f_3(x,u)}{\partial x_2}&\frac{\partial f_3(x,u)}{\partial x_3}&\frac{\partial f_3(x,u)}{\partial x_4}\\
\frac{\partial f_4(x,u)}{\partial x_1}&\frac{\partial f_4(x,u)}{\partial x_2}&\frac{\partial f_4(x,u)}{\partial x_3}&\frac{\partial f_4(x,u)}{\partial x_4}
\end{bmatrix}, \quad B = \begin{bmatrix}
\frac{\partial f_1(x,u)}{\partial u}\\\frac{\partial f_2(x,u)}{\partial u}\\\frac{\partial f_3(x,u)}{\partial u}\\\frac{\partial f_4(x,u)}{\partial u}
\end{bmatrix}
\end{equation}
And when we compute these derivatives, we obtain a linearized model in the form of state matrices
\begin{subequations}
	\begin{align}
		A &=\begin{bmatrix}0&1&0&0\\
				0&0&\frac{-gL_0l_1^2m_1^2}{(m_1L_0^2+I_0)(m_1l_1^2+I_1)-L_0^2l_1^2m_1^2}&0\\
				0&0&0&1\\
				0&0&\frac{gl_1m_1(m_1L_0^2+I_0)}{(m_1L_0^2+I_0)(m_1l_1^2+I_1)-L_0^2l_1^2m_1^2}&0
			\end{bmatrix}\\
		B &=	\begin{bmatrix}
				0\\ 
				\frac{m_1L_1^2+I_1}{(m_1L_0^2+I_0)(m_1l_1^2+I_1)-L_0^2l_1^2m_1^2}\\
				0\\
				\frac{-L_0l_1m_1}{(m_1L_0^2+I_0)(m_1l_1^2+I_1)-L_0^2l_1^2m_1^2}
			\end{bmatrix}\\
		C &= \begin{bmatrix}0&0&1&0\end{bmatrix}\\
		D &= 0
	\end{align}
\end{subequations}
Now those linearized equations of motion would be evaluated at two equilibrium positions: upright and downward. The reason is that at the downward position the system's output, which is the position of the pendulum, has a stable point at “$+\pi$” and “$-\pi$”, while at the upright position system has no stable point.
 
The model, obtained by linearization around the upright operation point, is used for fulfilling the main control objective, which is stabilizing the pendulum at the upright position. The second model is used to simulate process behavior during initial excitation by a Swing-up controller.
\section{Controller Synthesis}
In this section theory for the individual controllers is described. As we are aiming for swing-up control of the pendulum, several controllers should be designed. For the heuristic swing-up approach an energy shaping controller and a predictive controller. And for the optimal swing-up approach a nonlinear model predictive control strategy.
\subsection{Model Predictive Control}
MPC uses a model of the system to make predictions about the system’s future behavior. MPC solves an online optimization algorithm to find the optimal control action that drives the predicted output to the reference. MPC can handle MIMO systems that may have interactions between their inputs and outputs. It can also handle input and output constraints. MPC has preview capability; it can incorporate future reference information into the control problem to improve controller performance. Due to all these properties MPC provides the highest quality of control performance at the moment.
\subsubsection{MPC formulation}
The model predictive control requires the linear discrete-time state-space model of the process
\begin{subequations}\label{linmodel}
	\begin{align}	
	x(t+\Ts) = Ax(t) + Bu(t)\\
	y(t) = Cx(t) + Du(t)
	\end{align}
\end{subequations}
Thanks to the knowledge of that model, we can predict the evolution of states and outputs of the system.\\
Due to the setup of the controlled process, the MPC should be formulated to regulate the states of the system to the origin. And such MPC can be formulated as
\begin{subequations}
	\begin{align}
		\min_{u_0,...,u_{N-1}} &\sum_{k=0}^{N-1} \lrp{\left\| \ui{Q}{x}x_{t+k}\right\|_p+\left\|\ui{Q}{u}u_{t+k}\right\|_p}\\
	    \label{eq217b}s.t.\quad&x_{t+k+1} = Ax_{t+k} + Bu_{t+k}\quad  k \in \mathbb{N}_0^{N-1}\\
		&x_t = x(t)\\
		\label{cst_x}&x_{t+k}\in\mathcal{X}\qquad\qquad\qquad\qquad  k \in \mathbb{N}_0^{N-1}\\
		\label{cst_u}&u_{t+k}\in\mathcal{U}\qquad\qquad\qquad\qquad\,  k \in \mathbb{N}_0^{N-1}
	\end{align}
\end{subequations}
where $\mathcal{X}$ and $\mathcal{U}$ are polytopic state and input constraints respectively and defined as
\begin{subequations}
	\begin{align}
	\mathcal{X} &= \{\ui{H}{x}x\leq \ui{K}{x}\}\\
	\mathcal{U} &= \{\ui{H}{u}u\leq \ui{K}{u}\}
	\end{align}
\end{subequations}
Unfortunately, such formulation is inappropriate for quadratic programming solver, so it has to be reformulated in the form of a quadratic optimization problem.
\subsubsection{MPC as a QP optimisation problem}
The QP optimization problem has the following form
\begin{subequations}
	\begin{align}
	\min_{u_0,...,u_{N-1}} & z^\intercal Pz + 2Q^\intercal z + R\\
	\label{cst_qp}s.t.\quad&Hz\leq G\\
	&\ui{H}{eq}z = \ui{G}{eq}
	\end{align}
\end{subequations}
As the first step, the standard MPC cost function should be written in a vector form
\begin{equation}
	\min_{u_0,...,u_{N-1}} X^\intercal\ui{\tilde{Q}}{x}X + U^\intercal\ui{\tilde{Q}}{u}\,U\\
\end{equation}
where $X$ is a vector of predicted states, $U$ is an optimal trajectory of future control inputs and $\tilde{Q}_x$ and $\tilde{Q}_u$ are the matrices of original weight matrices.
\begin{equation}
	X = \begin{bmatrix}
	x_t\\x_{t+1}\\\vdots\\x_{t+N-1}
	\end{bmatrix}
\end{equation}
\begin{equation}
U = \begin{bmatrix}
u_t\\u_{t+1}\\\vdots\\u_{t+N-1}
\end{bmatrix}
\end{equation}
\begin{equation}
\tilde{Q}_x = \begin{bmatrix}
Q_x&0&\cdots&0\\
0&Q_x&\cdots&0\\
\vdots&\vdots&\ddots&\vdots\\
0&0&\cdots&Q_x
\end{bmatrix}
\end{equation}
\begin{equation}
\tilde{Q}_u = \begin{bmatrix}
Q_u&0&\cdots&0\\
0&Q_u&\cdots&0\\
\vdots&\vdots&\ddots&\vdots\\
0&0&\cdots&Q_u
\end{bmatrix}
\end{equation}
At the next step, the equality constrains (\ref{eq217b}) should be expressed in the vector form. We can achieve that by predicting the evolution of states over the whole prediction horizon.
\begin{equation}
\begin{split}
x_t &= x(t)\\
\hat{x}_{k+1} &= Ax_k + Bu_k\\
\hat{x}_{k+2} &= A\hat{x}_{k+1} + Bu_{k+1}\\
&= A^2x_k + ABu_k + Bu_{k+1}\\
\hat{x}_{k+3} &= A\hat{x}_{k+2} + Bu_{k+2}\\
&= A^3x_k + A^2Bu_k + ABu_{k+1} + Bu_{k+2}\\
&\vdots\\
\hat{x}_{k+N} &= A^Nx_k+\sum_{j=k}^{k+N-1}A^jBu_{k+N-j-1}
\end{split}
\end{equation}
Or, if we write these equations in a compact form, we obtain
\begin{equation}
	\begin{bmatrix}
	x_t\\\hat{x}_{t+1}\\ \hat{x}_{t+2}\\\vdots\\ \hat{x}_{t+N-1}
	\end{bmatrix} = 
	\begin{bmatrix}I\\A\\A^2\\ \vdots \\ A^{N-1}\end{bmatrix}x(t) + 
	\begin{bmatrix}
	0& 0&\cdots&0&0\\
	B&0&\cdots&0&0\\
	AB&B&\cdots&0&0\\
	\vdots&\vdots&\ddots&\vdots&\vdots\\
	A^{N-2}B&A^{N-3}B&\cdots&B&0\end{bmatrix}
	\begin{bmatrix}u_k\\u_{k+1}\\u_{k+2}\\\vdots\\u_{k+N-1}\end{bmatrix}
\end{equation}
Or in the short form
\begin{equation}\label{eq227}
	X = \tilde{A}x_t + \tilde{B}U
\end{equation}
At this point, we can substitute as $X$ in the cost function by (\ref{eq227}). Then we obtain the new objective function
\begin{equation}\label{eq228}
	\min_{u_0,...,u_{N-1}} (\tilde{A}x_t + \tilde{B}U)^\intercal\ui{\tilde{Q}}{x}(\tilde{A}x_t + \tilde{B}U) + U^\intercal\ui{\tilde{Q}}{u}\,U\\
\end{equation}
And when we expand and simplfie (\ref{eq228}), we obtain
\begin{equation}
	U^\intercal(\tilde{B}^\intercal\ui{\tilde{Q}}{x}\tilde{B} + \ui{\tilde{Q}}{u})U + 2x_t^\intercal\tilde{A}^\intercal\ui{\tilde{Q}}{x}\tilde{B}U + x_t^\intercal\tilde{A}\ui{\tilde{Q}}{x}\tilde{A}x_t
\end{equation}
Where we can cleary see matrices $P$, $Q$ and $R$, which occurs in a standart cost function for the quadratic optimization.
\begin{subequations}
	\begin{align}
		P &= \tilde{B}^\intercal\ui{\tilde{Q}}{x}\tilde{B} + \ui{\tilde{Q}}{u}\\
		Q &= (2x_t^\intercal\tilde{A}^\intercal\ui{\tilde{Q}}{x}\tilde{B}U)^\intercal\\
		R &= x_t^\intercal\tilde{A}\ui{\tilde{Q}}{x}\tilde{A}x_t
	\end{align}
\end{subequations}
Now only constrains remain. Constrains (\ref{cst_x}) and (\ref{cst_u}) must be reformulated as (\ref{cst_qp}). Those constrains we consider as a upper and lower bounds for the states and control inputs respectively
\begin{subequations}
	\begin{align}
		\ui{x}{min} &\leq x_k \leq \ui{x}{max} \quad k \in \mathbb{N}_0^{N-1}\\
		\ui{u}{min} &\leq u_k \leq \ui{u}{max} \quad k \in \mathbb{N}_0^{N-1}
	\end{align}
\end{subequations}
Now we split and vectorise those constrains
\begin{subequations}
\begin{align}
	X &\leq \ \; \, \ui{X}{max}\\
	-X &\leq -\ui{X}{min}\\
	U &\leq \ \; \, \ui{U}{max}\\
	-U &\leq -\ui{U}{min}
\end{align}
\end{subequations}
In the next step, we substitute $X$ in the states constrains by a (\ref{eq227}) and express $U$
\begin{subequations}
	\begin{align}
	\tilde{B}\,U &\leq \ \; \,\ui{X}{max} - \tilde{A}x_t\\
	-\tilde{B}\,U &\leq -\ui{X}{min} + \tilde{A}x_t\\
	U &\leq \ \; \,\ui{U}{max}\\
	-U &\leq -\ui{U}{min}
	\end{align}
\end{subequations}
Those constraints are in the standard form and by combining them, we obtain matrices of inequality constraints $H$ and $G$ for the quadratic programming problem
\begin{equation}
	H = \begin{bmatrix}
	\ \; \,\tilde{B}\\
	-\tilde{B}\\
	\ \ \, I\\
	-I\\
	\end{bmatrix}, \quad
	G = \begin{bmatrix}
	\ \; \,\ui{X}{max} - \tilde{A}x_t\\
	-\ui{X}{min} + \tilde{A}x_t\\
	\ \ \:\ui{U}{max}\\
	-\ui{U}{min}
	\end{bmatrix}
\end{equation}
At this point, as matrices $R$, $Q$, $R$, $H$ and $G$ are defined, we can implement MPC by using any quadratic programming solver.
\subsection{Energy Shaping Controller}
For the initial excitation of the system, we use the energy-based swing-up controller. The strategy with this controller is that we increase the amplitude of swings by increasing the energy of the system with every swing. The energy is added by controlling arms movements and depends on the actual energy of the pendulum. The actual energy of the pendulum can be calculated from the actual position of the pendulum and its velocity: 
\begin{equation}
E = \frac{m_1gl_1}{2}\lrp{\lrp{\frac{\dot{\theta_1}}{\omega_0}}^2+\cos\theta_1 - 1}
\end{equation}
Than the control law has following form:
\begin{equation}
	u = \ui{k}{v}E\,\mathrm{sign}\lrp{\dot{\theta_1}\cos\theta_1}
\end{equation}
Where element $\mathrm{sign}\lrp{\dot{\theta_1}\cos\theta_1}$ determines direction i which the force will be applied and $k_vE$ is the gain of the controller.
\newpage
\subsection{Nonlinear Model Predictive Control}
In general to predict bechavior of the system MPC uses a linear predictive model (\ref{linmodel}). Though the original dynamic model of Furuta pendulum is nonlinear (\ref{nonlinmodel}) and such linearised model is precise only in short range around linearisation point. And as the main task is to perform swing-up control of the pendulum with the stabilization at the upright position via using only one controller, another control strategy should be designed.
\subsubsection{Nonlinear Programming Formulation}
To design an NMPC strategy, a Nonlinear Programming (NLP) problem should be formulated
\begin{subequations}\label{NLP}
	\begin{align}
	\min_{u_0,...,u_{N-1}} &\sum_{k=0}^{N-1} (\left\| \ui{Q}{x}x_{t+k}\right\|_p+\left\|\ui{Q}{u}u_{t+k}\right\|_p)\\
	s.t.\quad&x_{t+k+1} = \phi(x_{t+k},u_{t+k})\quad k \in \mathbb{N}_0^{N-1}\\
	&x_t = x(t)\\
	&x_{t+k}\in\mathcal{X}\qquad\qquad\qquad\quad\!\: k \in \mathbb{N}_0^{N-1}\\
	&u_{t+k}\in\mathcal{U}\qquad\qquad\qquad\quad\, k \in \mathbb{N}_0^{N-1}
	\end{align}
\end{subequations}
where $\phi(x_{t+k},u_{t+k})$ is a numerical integration operator that solves the following initial value problem (IVP) and returns the solution at $t_{k+1}$.
\begin{equation}
0=\ui{f}{impl}\lrp{\dot{x}(t), x(t),u(t),t},\quad x(0)=x_k.
\end{equation}
To solve such nonlinearly constrained optimization problem Sequential Quadratic Programming (SQP) algorithm will be used.
\subsubsection{Sequential Quadratic Programming}
SQP is the most successful method for solving nonlinearly constrained optimization problems. The basic idea of SQP is to model 
NLP at a given iterative $x^i$ as a Quadratic Programming (QP) subproblem, then the solulion of that QP subproblem is used to construct  a new iterate $x^{i+1}$.  And hopefully at the some point will converge to a solution $x^{*}$. This subproblem is assumed to reflect the local properties of the original problem.\\
For instance at iteration \textit{i} problem (\ref{NLP}) is linearized and  the following quadratic programming (QP) problem constucted:
\begin{equation}\label{QP}
\begin{aligned}
\min_{\Delta \mathbf{x},\Delta \mathbf{u}} \quad & \sum_{k=0}^{N-1}( \frac{1}{2}
\begin{bmatrix}
\Delta x_k\\
\Delta u_k
\end{bmatrix}^\intercal \begin{bmatrix}
Q_k^i & S_k^i \\
S_k^{i^\top} & R_k^i
\end{bmatrix}
\begin{bmatrix}
\Delta x_k\\
\Delta u_k
\end{bmatrix} + \begin{bmatrix}
g_{x_k}^i\\
g_{u_k}^i
\end{bmatrix}^\intercal
\begin{bmatrix}
\Delta x_k\\
\Delta u_k
\end{bmatrix} ) \\
s.t. \quad & \Delta x_0=\hat{x}_0-x_0,\\
& \Delta x_{k+1}=A_{k}^i \Delta x_{k}+ B_{k}^i \Delta u_{k} +a_{k}^i, \,\forall k \in \{0,...,N-1\}\\
& \underline{x}_k - x_k^i\leq \Delta x_k\leq \overline{x}_k-x_k^i,\,\forall k \in \{1,...,N\}\\
& \underline{u}_k - u_k^i\leq \Delta u_k\leq\overline{u}_k-u_k^i,\,\forall k \in \{0,...,N-1\}\\
\end{aligned}
\end{equation}
where 
\begin{equation}
\begin{aligned}
&\Delta \mathbf{x}=\mathbf{x}-\mathbf{x}^i,\\
&\Delta \mathbf{u}=\mathbf{u}-\mathbf{u}^i
\end{aligned}
\end{equation}
and
\begin{equation}\label{QP data}
\begin{aligned}
&g_{x_k}^i = \frac{\partial d^i}{\partial x_k},\quad g_{u_k}^i = \frac{\partial d^i}{\partial u_k},\\
&A_k^i=\frac{\partial \phi_k}{\partial x_k}, \quad B_k^i=\frac{\partial \phi_k}{\partial u_k},\quad a_k^i = \phi(x_k^i,u_k^i)-x_{k+1}^i,\\
\end{aligned}
\end{equation}
Where outer objective function $d:\mathbb{R}^{n_y}\rightarrow \mathbb{R}$, is convex. The Hessian matrices $Q_k,S_k,R_k$ can be approximated by the Gauss-Newton (GN) or the Generalized-Gauss-Newton (GGN) method. By solving that QP the optimal primal solution $(\Delta \mathbf{x}^{i^*}, \Delta \mathbf{u}^{i^*})$ is obtained. That primal solution is used to update the solution of \eqref{NLP} by
\begin{subequations}
	\begin{align}
\mathbf{x}^{i+1} = \mathbf{x}^{i} + \alpha^i \Delta\mathbf{x}^{i^*}, \\ \mathbf{u}^{i+1} = \mathbf{u}^{i} + \alpha^i \Delta\mathbf{u}^{i^*},
\end{align}
\end{subequations}
where $\alpha^i$ is the step length determined by globalization strategies. 
\chapter{Simulation Results}
\section{Model Predictive Controller}
Due to the pendulums setup, the MPC formulation (\ref{mpcgeneral}) could be applied directly. As known, a linear MPC uses a linear prediction model of the process to predict the behavior of the system over the prediction horizon. So such a predictive controller is used to control the pendulum in range, where such a linear predictive model is relevant.So the first step of the construction of such a controller is to obtain the discrete-time linear predictive model via discretization of (\ref{linmatrices}) at the upright operation point 
\begin{equation}
	\ui{x}{up} = \begin{bmatrix}0,0,0,0\end{bmatrix}^\intercal, 
\end{equation}
with the discretization step $0.02$, what is equal to real devices sampling time. 
\begin{subequations}\label{dismatrices}
	\begin{align}
	\ui{A}{D} &=\begin{bmatrix}
	1&0.02&-0.0031& 0\\
	0&1&-0.3140&-0.0031\\
	0&0&\ \; \,1.0161&\ \; \,0.0201\\
	0&0&\ \; \,1.6096&\ \; \,1.0161
	\end{bmatrix}\\
	\ui{B}{D} &=\begin{bmatrix}
	\ \; \,0.0035\\
	\ \; \,0.3509\\
	-0.0081\\
	-0.8083
	\end{bmatrix}\\
	\ui{C}{D} &= \begin{bmatrix}0&0&1&0\end{bmatrix}\\
	\ui{D}{D} &= 0
	\end{align}
\end{subequations}
The next step is to design weight matrices $\ui{Q}{x}$ and $\ui{Q}{u}$.
\begin{equation}
\ui{Q}{x} = \begin{bmatrix}
1.5&0&0&0\\
0&0.08&0&0\\
0&0&10&0\\
0&0&0&0.2
\end{bmatrix}, \quad \ui{Q}{u} = 1
\end{equation}
The main weights are given obviously to the position of the pendulum, what is our main controlled parameter, and the position of the arm, to prevent the possible scenario when the pendulum is stabilized at the upright position and the arm rotates constantly. 
The remain MPC parametres are shown in the following table
\begin{table}[h]
	\caption{MPC parameters during the stabilization at the upright position}
	\begin{tabular}{l c c}
		\noalign{\hrule height 1pt}
		Parameter&Symbol&Value\\
		\hline
		Prediction horizon&$N$&$20$\\
		Initial condition&$x_0$&$\begin{bmatrix}-1,-2,0.5,2\end{bmatrix}^\intercal$\\
		Constraint on control input-upper bound&$\ui{u}{max}$&$\ \; \,5$\\
		Constraint on control input-lower bound&$\ui{u}{min}$&$-5$\\
		\noalign{\hrule height 1pt}
	\end{tabular}
\end{table}
\newpage
\begin{figure}[H]
	\centering
	\includegraphics[width=1.1\linewidth]{images/MPC}
	\caption{Simulation results for a scenario with stabilization at the upright position by linear MPC. The plots depict, respectively, the individual states , the third of which is the controlled pendulum position $\theta_1(t)$, and the control input $\tau(t)$.}
	\label{mpc}
\end{figure}
\newpage
\section{Energy-Shaping Controller}
The main purpose of the Energy-Shaping controller is to swing the pendulum from the downside position into the upright position where the control of the process will be taken by another controller. 
Such controller can be designed by applying the control law (\ref{energy-shaping}) directly as a state-feedback controller.
\newpage
\begin{figure}[H]
	\centering
	\includegraphics[width=1.1\linewidth]{images/Swing}
	\caption{Initial oscillation of the system by the Energy-Shaping controller. The plots depict, respectively, the individual states , the third of which is the controlled pendulum position $\theta_1(t)$, and the control input $\tau(t)$.}
	\label{swing}
\end{figure}
\newpage
\section{Heuristic Swing-Up Control}
To perform Swing-Up Control of the pendulum, a combined control strategy should be designed, because no MPC nor Swing-Up controller can not perform a Swing-Up control by itself. So the strategy of Heuristic Swing-Up Control is that at the beginning the pendulum is steady at the downside position and the Energy-Shaping controller is used to add the energy to the pendulum via its oscillation. As more energy is added to the pendulum, the greater the oscillations become. As the pendulum is close to the upright operation point, the control law switches from Energy-Shaping to MPC. And MPC finishes the Swing-Up Control by stabilizing the pendulum at the upright position.
\newpage
\begin{figure}[H]
	\centering
	\includegraphics[width=1.1\linewidth]{images/MPC-LQR_Swing}
	\caption{Simulation result of the Swing-up control of the pendulum by Energy-Shaping+Mpc combined control strategy. The first plot depict the controlled pendulum position $\theta_1(t)$, and the second - control input $\tau(t)$.}
	\label{combo}
\end{figure}
\newpage
\section{NMPC}
To design a Nnlinear Model Predictive Controller freely availabe \textit{MATMPC} toolbox is used. This toolbox which could be downloaded from \textit{https://github.com/chenyutao36/MATMPC}. Also additional software is required:
\begin{itemize}
	\item \textbf{CasAdi} - the state-of-the-art automatic/algorithmic differentiation toolbox.
	\item \textbf{MinGW-w64 C/C++ Compiler} - algorithmic routines are compiled
	into MEX functions using this compiler.
\end{itemize}
Basically \tt{MATMPC} solves the folloving NLP problem:
\begin{subequations}\label{NLP}
	\begin{align}
	\min_{U} &\sum_{k=0}^{N-1} (\left\| \ui{Q}{x}x_{k}\right\|_p+\left\|\ui{Q}{u}u_{k}\right\|_p)\\
	s.t.\quad&x_{k+1} = \phi(x_{k},u_{k})\qquad k \in \mathbb{N}_0^{N-1}\\
	&x_0 = x(t)\\
	&x_{k}\in\mathcal{X}\qquad\qquad\qquad\!\: k \in \mathbb{N}_0^{N-1}\\
	&u_{k}\in\mathcal{U}\qquad\qquad\qquad\, k \in \mathbb{N}_0^{N-1}
	\end{align}
\end{subequations}
where $\phi(x_{t+k},u_{t+k})$ is a numerical integration operator that solves the following initial value problem (IVP) and returns the solution at $t_{k+1}$.
\begin{equation}
0=\ui{f}{impl}\lrp{\dot{x}(t), x(t),u(t),t},\quad x(0)=x_k.
\end{equation}
To obtain a numerical solution to that NLP problim \tt{MATMPC} uses a SQP strategy to construct and solve a sequence of following quadratic problems
\begin{equation}\label{QP}
\begin{aligned}
\min_{\Delta \mathbf{x},\Delta \mathbf{u}} \quad & \sum_{k=0}^{N-1}( \frac{1}{2}
\begin{bmatrix}
\Delta x_k\\
\Delta u_k
\end{bmatrix}^\intercal \begin{bmatrix}
Q_k^i & S_k^i \\
S_k^{i^\top} & R_k^i
\end{bmatrix}
\begin{bmatrix}
\Delta x_k\\
\Delta u_k
\end{bmatrix} + \begin{bmatrix}
g_{x_k}^i\\
g_{u_k}^i
\end{bmatrix}^\intercal
\begin{bmatrix}
\Delta x_k\\
\Delta u_k
\end{bmatrix} ) \\
s.t. \quad & \Delta x_0=\hat{x}_0-x_0\\
& \Delta x_{k+1}=A_{k}^i \Delta x_{k}+ B_{k}^i \Delta u_{k} +a_{k}^i \qquad k \in \mathbb{N}_0^{N-1}\\
& \Delta x_{min}\leq \Delta x_k\leq \Delta x_{max}\qquad\qquad\quad\; k \in \mathbb{N}_1^{N}\\
& \Delta u_{min}\leq \Delta u_k\leq\Delta u_{max}\qquad\qquad\quad\; k \in \mathbb{N}_0^{N-1}\\
\end{aligned}
\end{equation}
where 
\begin{equation}
\begin{aligned}
&\Delta \mathbf{x}=\mathbf{x}-\mathbf{x}^i\\
&\Delta \mathbf{u}=\mathbf{u}-\mathbf{u}^i
\end{aligned}
\end{equation}
and
\begin{equation}\label{QP data}
\begin{aligned}
&g_{x_k}^i = \frac{\partial d^i}{\partial x_k},\quad g_{u_k}^i = \frac{\partial d^i}{\partial u_k},\\
&A_k^i=\frac{\partial \phi_k}{\partial x_k}, \quad B_k^i=\frac{\partial \phi_k}{\partial u_k},\quad a_k^i = \phi(x_k^i,u_k^i)-x_{k+1}^i,\\
\end{aligned}
\end{equation}
Where outer objective function $d:\mathbb{R}^{n_y}\rightarrow \mathbb{R}$, is convex. The Hessian matrices $Q_k,S_k,R_k$ can be approximated by the Gauss-Newton (GN) or the Generalized-Gauss-Newton (GGN) method. By solving that QP the optimal primal solution $(\Delta \mathbf{x}^{i^*}, \Delta \mathbf{u}^{i^*})$ is obtained. That primal solution is used to update the solution of \eqref{NLP} by
\begin{subequations}
	\begin{align}
	\mathbf{x}^{i+1} = \mathbf{x}^{i} + \alpha^i \Delta\mathbf{x}^{i^*}, \\ \mathbf{u}^{i+1} = \mathbf{u}^{i} + \alpha^i \Delta\mathbf{u}^{i^*},
	\end{align}
\end{subequations}
where $\alpha^i$ is the step length determined by globalization strategies. 
Now only weight matrices $Q_x$ and $Q_u$ remain:
\begin{equation}
	\ui{Q}{x} = \begin{bmatrix}
	1&0&0&0\\
	0&0.1&0&0\\
	0&0&2&0\\
	0&0&0&0.1
	\end{bmatrix} \quad \ui{Q}{u} = 0.1
\end{equation}
And now with such setup, a swing-up control of the Furuta Pendulum is performed
\newpage
\begin{figure}[H]
	\centering
	\includegraphics[width=1.1\linewidth]{images/NMPC}
	\caption{Simulation result of the Swing-up control of the pendulum by NMPC strategy.The plots depict, respectively, the individual states , the third of which is the controlled pendulum position $\theta_1(t)$, and the control input $\tau(t)$.}
	\label{NMPC}
\end{figure}
% conclusions
\chapter{Conclusions}
The goal of this thesis was to design a set of controllers for the underactuated fast and extremely non-linear process, which was Furuta's pendulum. For that purpose Swing-Up, LQR, MPC and NMPC controllers were designed.\\
In the beginning, we got the nonlinear dynamic model of the process. But only NMPC controller can control the process, which is described by such model. To make it suitable for Swing-Up, LQR and MPC controllers we had to linearise it relative to two operating points: at the downside position, where the process is stable, and at the upright position, where the process is unstable. The reason for that is that the process is controlled by LQR and MPC controllers when the pendulum is in the upper position, and it is controlled by a Swing-Up controller when the pendulum is in the lower position.\\
All controllers were designed using MATLAB and also tested in simulations. By evaluating simulation results we can conclude that all controllers were designed properly, and could be used to control the real process. 


% Appendices (Prílohy) comment by "%" if not neccesary
\appendix
%\chapter{Resumé}
\label{ch:resume}

Resumé v slovenčine, sa píše v prípade, že záverečná práca ja napísaná v 
anglickom jazyku. Rozsah resumé tvorí 5-10\% rozsahu diplomovej práce.

%----------------------------------------------------------------%
%  The Backmatter !! Do NOT change the structure!!               %
%----------------------------------------------------------------%
% Bibliography to TOC
% do not remove
\backmatter
\providebibliography
\bibliography{bibfile}

%----------------------------------------------------------------%
%   The end of the document                                      %
%----------------------------------------------------------------%
\end{document}
