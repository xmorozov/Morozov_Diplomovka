\chapter{Introduction}
\label{ch:intro}

This \LaTeX template is intended for student writing their final theses at 
Faculty of Chemical and Food Technology. This chapter covers the basic setup of 
the LaTeX template as well as simple guidelines for high-quality typesetting. 
This template generates the thesis structure based on requirements. The 
template is intended for Slovak-writing as well as for English-writing students
\begin{verbatim}
\documentclass[english]{iamthesis}
% changle "english" to "slovak" for english version of the thesis
\end{verbatim}

The file structure of this LaTeX template
\begin{verbatim}
	|-- content/
	   |-- abstract_en.tex
	   |-- abstract_sk.tex
	   |-- introduction.tex	   
	   |-- assignment.pdf
	   |-- theory.tex
	   |-- conclusions.tex
	   |-- resume.tex
	|-- images/
		 |-- fchpt_logo_color.pdf
	|-- main.tex
	|-- bibfile.bib
\end{verbatim}

The root document is the \texttt{main.tex}. Here, the student must specify 
key-values of the thesis (lines 7 to 32), namely:
\begin{verbatim}
% student name with title e.g. Ing. Martin Klaučo
\def\thesisauthor{Students Name} 

% year of submmiting to AIS
\def\thesisyear{Year}

% registration number generated by AIS e.g. 19990-50920
\def\thesisnumber{Number} 

% thesis type: BACHELOR|MASTER|DISSERTATION or in slovak 
% BAKALÁRSKA|DIPLOMOVÁ|DIZERTAČNÁ
\def\thesistype{DIPLOMOVÁ}

% thesis title
\def\thesistitle{Title of the Thesis}

% thesis supervisor including degrees e.g. MSc. Ing. Martin Klaučo, PhD.
\def\thesissupervisor{Ing. Martin Klaučo, PhD.}

% study field (translate to english if neccesarry) e.g. 
% "Riadenie Procesov" or "Process Control"
\def\thesisprogram{Riadenie Procesov}

% Institute (translate to english if neccesary)
% e.g., "Institute of Information Engineering, Automation, and Mathematics"
\def\thesisinst{Ústav Automatizácie, Informatizácie a Riadenia Procesov}
\end{verbatim}

The file \texttt{assignment.pdf} is to be replaced by assignment generated by 
the AIS system.

The file \texttt{bibfile.bib} is to be expanded with your references. You can 
also provide your own bibfile. See command 
\texttt{\textbackslash{providebibliography}}.

\section{Typesetting Guidelines}
This template comes with predefined macros, help to ease up the typesetting of 
math. Refer to table~\ref{tab:commands} for examples.


Before submitting the thesis check if
\begin{itemize}
	\item all equations are referenced in the text (remember, you can refer only 
	to equation which was already written),
	\item all figures are referenced in the text (remember, figure must not 
	appear necessarily on the same page as you would like),
	\item all tables are referenced in the text (remember, figure must not 
		appear necessarily on the same page as you would like),
	\item every chapter, section or subsection must start with a paragraph,
	\item every chapter, section or subsection must end with text, not with 
	equation, nor with table or figure,
	\item if writing in English, check you grammar with free tool 
	\url{www.grammarly.com}.
\end{itemize}


\begin{table}[h]
	\centering
	\caption{Example of built-in commands}
	\label{tab:commands}
	\begin{tabular}{llp{5cm}}
		\toprule
			command & results & remark\\
		\midrule
		  \texttt{\textbackslash{ui\{F\}\{a\}}} &  $\ui{F}{a}$ & is used 
		  when the subindex ''a'' is part 
		  of notation\\
		  \texttt{\textbackslash{uis\{F\}\{a\}\{k\}}} &  $\uis{F}{a}{k}$ & is 
		  used  when  the  subindex ''a'' is  part  of notation and ''k'' is 
		  variable 
		  \\
		  \texttt{\textbackslash{Ts}} & $\Ts$ & sampling time \\[2pt]
		  \texttt{\textbackslash{lrp\{ \textbackslash{frac\{a\}\{b\}} \}}} & 
		  $\displaystyle\lrp{\frac{a}{b}}$& 
		  expandable parenthesis based on 
		  the expression size\\
		  \texttt{\textbackslash{lrb\{ \textbackslash{frac\{a\}\{b\}} \}}} & 
		  $\displaystyle\lrb{\frac{a}{b}}$& 
		  expandable brackets based on 
		  the expression size\\
		  \texttt{\textbackslash{dif\{f(x)\}\{x\}}} & 
		  $\displaystyle\dif{f(x)}{x}$& \\[10pt]
		  \texttt{\textbackslash{diff\{f(x, y)\}\{x\}}} & 
		  $\displaystyle\diff{f(x,y )}{x}$& \\[10pt]
		  \texttt{\textbackslash{diffxy\{f(x,y)\}\{x\}\{y\}}} & 
		  $\displaystyle\diffxy{f(x, y)}{x}{y}$& \\[10pt]
		  \texttt{\textbackslash{diffat\{f(x,y)\}\{x\}\{x=0\}}} & 
		  $\displaystyle\diffat{f(x, y)}{x}{x=0}$& \\[10pt]
		  \texttt{\textbackslash{difat\{f(x,y)\}\{x\}\{x=0\}}} & 
		  $\displaystyle\difat{f(x, y)}{x}{x=0}$& \\[10pt]
		  \texttt{\textbackslash{dt\{f(t)\} } } & $\displaystyle\dt{f(t)}$  & \\
		\bottomrule
	\end{tabular}
	% the "\displaystyle" command for increased font size of the math expression 
	%  in table
\end{table}

Example of citing literature~\cite{boyd:book:2009:cvx}.