\chapter{Theoretical Foundations}
This part of the thesis dedicated to the theoretical foundations, necessary to control the pendulum. In the beginning, the pendulum device is described and its dynamic model is derived. Then the main control strategies and the theory for the individual controllers are discussed. There are two control strategies we are aiming for the heuristic swing-up and the optimal swing-up control strategies. Also, several controllers must be described, as in the heuristic swing-up approach an energy shaping controller and a predictive controller are used and in the optimal swing-up approach a nonlinear model predictive controller is used.
\section{Furuta's Pendulum}\label{furuta_Theory}
In this section, the operation of the Furuta pendulum device will be explained, and then two dynamic models linear and non-linear will be derived.
\subsection{Device Overview}
Rotational inverted pendulum was first developed in 1992 at the Tokyo Institute of Technology by Katsuhisa Furuta as an example of a complex nonlinear oscillator in the need of control algorithms testing in process control theory.
The pendulum composes of motor-driven arm, which rotates in the horizontal plane and a pendulum, attached to that arm, which freely rotates in the vertical plane. The system is underactuated and extremely nonlinear due to the gravitational forces and the coupling arising from the Coriolis and centripetal forces. The schematic representation of the pendulum is shown in \ref{furuta:schematic}.
\newpage
\begin{figure}[h]
	\centering
	\includegraphics[width=.45\linewidth]{images/furuta}
	\caption{Scematic Representation of Furuta's Pendulum device}
	\label{furuta:schematic}
\end{figure}
The parameters from Fig.~\ref{furuta:schematic} are explained in the following table.
\begin{table}[H]
	\centering
	\caption{Parametres of Furuta pendulum}
	\label{furuta:params}
	\begin{tabular}
		{l l l}
		\noalign{\hrule height 1pt}
		Symbol&Parameter&Unit\\
		\noalign{\hrule height 1pt}
		$g$&gravitational acceleration&\si{\metre\per\square\second}\\
		$m_0$&mass of arm&\si{\kilogram}\\
		$m_1$&mass of pendulum&\si{\kilogram}\\
		$L_0$&length of arm&\si{\metre}\\
		$L_1$&length of pendulum&\si{\metre}\\
		$l_0$&location of the arms center of mass&\si{\metre}\\
		$l_1$&location of the pendulums center of mass&\si{\metre}\\
		$I_0$&moment of inertia of arm&\si{\kilogram\per\square\metre}\\
		$I_1$&moment of inertia of pendulum&\si{\kilogram\per\square\metre}\\
		$\theta_0$&arm angle&\si{\radian}\\
		$\theta_1$&pendulum angle&\si{\radian}\\
		$\tau$&motor torque&\si{\volt}\\
		\hline
	\end{tabular}
\end{table}
So, we consider the rotational inverted pendulum mounted to a DC motor as shown in (\ref{furuta:schematic}). The DC motor is used to apply a torque $\tau$ to arm. The link between arm and pendulum is not actuated but free to rotate. The two arms have lengths $L_0$ and $L_1$. The arm and pendulum have masses $m_1$ and $m_2$ which are located at $l_0$ and $l_1$, respectively, which are the lengths from the point of rotation of the arm to its center of mass. The arm and the pendulum have inertia tensors $J_0$ and $J_1$ (about the centre of mass of the arm). The angular rotation of arm is measured in the horizontal plane where a counterclockwise direction (when viewed from above) is positive. The angular rotation of pendulum is measured in the vertical plane where a counterclockwise direction (when viewed from the front) is positive.
\subsection{Non-linear Dynamic Model}
To design a predictive controller, the knowledge of the dynamic model of the process is necessary. The most precise way to describe the process dynamic is by the non-Linear dynamic model. The analytical model of Furuta pendulum, based on the equations of motion could be derived using the Lagrangian formulation of the system dynamics of the mechanical system ~\cite{furuta:model}, which represent the most commonly used method for establishing equations of motion of difficult mechanical systems.

The Lagrangian defined as the difference in kinetic and potential energies
\begin{equation}
L = \ui{E}{k} - \ui{E}{p}.
\end{equation}
As we know, the kinetic energy is a function of both the speed and position and the potential energy is a function of position. Therefore to define the energies, a way to describe the state of the point during the rotary motion is required.
\subsubsection{Rotation Matrices}
Rotation matrices are used to describe rotation in a Euclidean space. For three-dimensional Cartesian coordinate system two rotation matrices are defined.
The rotation matrix from the base to arm is
\begin{equation}
	R_0 = 	\begin{bmatrix}
					\ \ \,\cos(\theta_0) & \sin(\theta_0) & 0\\
					-\sin(\theta_0) & \cos(\theta_0) & 0\\
					0 & 0 & 1
				\end{bmatrix}.
\end{equation}
The rotation matrix from arm to pendulum is derived by
initially applying a diagonal matrix to that maps the frame
0 to frame 1, followed by a rotation matrix for $\theta_1$, given by
\begin{equation}
R_1 = 	\begin{bmatrix}
			0 & \sin(\theta_1) & -\cos(\theta_1)\\
			0 & \cos(\theta_1) & \ \ \,\sin(\theta_1)\\
			1 & 0 & 0                      
		\end{bmatrix}.
\end{equation}
These rotation matrices are necessary to define the velocities of the arm and the pendulum during the rotational motion. 
\subsubsection{Velocities}
The angular velocity of the arm is given by
\begin{equation}
	\omega_0 = 	\begin{bmatrix}
					0 & 0 & \dot{\theta}_0
				\end{bmatrix}^\intercal.
\end{equation}
Let the velocity of the base frame be at rest, such that the joint
between the frame and the arm is also at rest, that is,
\begin{equation}
	\text{\textbf{v}}_0 = 	\begin{bmatrix}
					0 & 0 & 0
			\end{bmatrix}^\intercal.
\end{equation}
The total linear velocity of the centre of mass of arm is
given by
\begin{equation}
	\text{\textbf{v}}_{0, \text{c}} = v_0+\omega_0\times\begin{bmatrix}
					l_0 & 0 & 0
				\end{bmatrix}^\intercal = \begin{bmatrix}
				0 & \dot{\theta}_0l_0 & 0
				\end{bmatrix}^\intercal.
\end{equation}
The angular velocity of the pendulum is given by
\begin{equation}
\omega_1 = R_1\omega_0 + \begin{bmatrix}
				0 & 0 & \dot{\theta}_1
			\end{bmatrix}^\intercal = 
			\begin{bmatrix}
				-\cos(\theta_1)\dot{\theta}_0 & \sin(\theta_1)\dot{\theta}_0 & \dot{\theta}_1
			\end{bmatrix}^\intercal.
\end{equation}
The velocity of the joint between arm and the pendulum in
reference frame 1 is
\begin{equation}
	\text{\textbf{v}}_{1} = R_1\lrp{\omega_0\times	\begin{bmatrix}
									L_0 & 0 & 0
									\end{bmatrix}^\intercal} = 
	\begin{bmatrix}
		\dot{\theta}_0L_0\sin(\theta_1)\\
		\dot{\theta}_0L_0\cos(\theta_1)\\
		0
	\end{bmatrix}.
\end{equation}
The total linear velocity of the centre of mass of arm is
given by
\begin{equation}
	\text{\textbf{v}}_{1,\text{c}} = \text{\textbf{v}}_1+\omega_1\times
		\begin{bmatrix}
			l_1 & 0& 0
		\end{bmatrix}^\intercal=
	\begin{bmatrix}
		\dot{\theta}_0L_0\sin(\theta_1)\\
		\dot{\theta}_0L_0\cos(\theta_1) + \dot{\theta}_1l_1\\
		-\dot{\theta}_0l_1\sin(\theta_1)
	\end{bmatrix}.														
\end{equation}
As the velocities are defined, the next step is to define the energies of the system.
\subsubsection{Inertia Tensors}
The moment of inertia is a quantity that determines the torque needed for a desired angular acceleration about a rotational axis. As the motion of Furuta pendulum is a 3D process, the coordinate systems of arm and pendulum are the three-dimensional so that the inertia tensors have the form of diagonal matrices
\begin{subequations}
	\begin{align}
	\text{\textbf{J}}_0 = 	\begin{bmatrix}
	J_{0,\text{xx}} & 0 & 0\\
	0 & J_{0,\text{yy}} & 0\\
	0 & 0 & J_{0,\text{zz}}\\
	\end{bmatrix},\\
	\text{\textbf{J}}_1 = 	\begin{bmatrix}
	J_{1,\text{xx}} & 0 & 0\\
	0 & J_{1,\text{yy}} & 0\\
	0 & 0 & J_{1,\text{zz}}\\
	\end{bmatrix}.
	\end{align}
\end{subequations}
Where the diagonal elements ($J_{\text{xx}},J_{\text{yy}},J_{\text{zz}}$) represent the singular moment of inertia around the corresponding axis when the objects are rotated around that axis. In the nex step those matrices are used to define the energies of the system.
\subsubsection{Energies}
The potential energy of arm is
\begin{equation}
	\ui{E}{p,0} = 0,
\end{equation}
and the kinetic energy is
\begin{equation}
	\ui{E}{k,0} = \frac{1}{2}\lrp{\text{\textbf{v}}_{0,\text{c}}^\intercal m_0\textbf{v}_{0,\text{c}} + \omega_0^\intercal \text{\textbf{J}}_0\omega_0 } = 
	\frac{1}{2}\dot{\theta}_0^2\lrp{ m_0l_0^2+J_{0,\text{zz}} }.
\end{equation}
The potential energy of the pendulum is
\begin{equation}
\ui{E}{p,1} = gm_1l_1\lrp{1-\cos(\theta_1)},
\end{equation}
and the kinetic energy is
\begin{equation}
\begin{split}
	\ui{E}{k,1} = &\frac{1}{2}\lrp{\text{\textbf{v}}_{1,\text{c}}^\intercal m_1\textbf{v}_{1,\text{c}} + \omega_1^\intercal \text{\textbf{J}}_1\omega_1 }\\
	= &\frac{1}{2}\dot{\theta}_0^2\lrp{m_1L_1^2+\lrp{m_1l_1^2+J_{1,\text{yy}}}\sin^2(\theta_1)+J_{1,\text{xx}}\cos^2(\theta_1)}\\
	&+\frac{1}{2}\dot{\theta}_1^2\lrp{ J_{1,\text{zz}}+m_1l_1^2 }+m_1L_0l_1\cos(\theta_1)\dot{\theta}_0\dot{\theta}_1.
\end{split}
\end{equation}
The total potential and kinetic energies are given, respectively, by
\begin{subequations}
	\begin{align}
		\ui{E}{p} &= \ui{E}{p,0} + \ui{E}{p,1}, \\
		\ui{E}{k} &= \ui{E}{k,0} + \ui{E}{k,1}.
	\end{align}
\end{subequations}
Now, all necessary prerequisite equations for the derivation of the Lagrange equations are presented.
\subsubsection{Euler-Lagrange Equations}
For two degrees of freedom system the Euler-Lagrange equations have the following form
\begin{subequations}\label{ELequation}
	\begin{align}
		\dif{}{t}\lrp{\diff{L}{\dot{\theta}_0}}-\diff{L}{\dot{\theta}_0} = \tau,\\
		\dif{}{t}\lrp{\diff{L}{\dot{\theta}_1}}-\diff{L}{\dot{\theta}_1} = 0.
	\end{align}
\end{subequations}
Also, a few simplifications could be made. Furuta pendulum has long slender arm and pendulum, such that the moment of inertia along the axis of the arm and the pendulum is negligible. In addition, arm and pendulum have rotational symmetry, such that the moments of inertia in two of the principal axes are equal. Thus, the inertia tensors may be approximated as follows
\begin{subequations}
	\begin{align}
	\text{\textbf{J}}_0 = 	\begin{bmatrix}
	J_{0,\text{xx}} & 0 & 0\\
	0 & J_{0,\text{yy}} & 0\\
	0 & 0 & J_{0,\text{zz}}\\
	\end{bmatrix} = \begin{bmatrix}
		0 & 0 & 0\\
	0 & J_0 & 0\\
	0 & 0 & J_0\\
	\end{bmatrix},\\
	\text{\textbf{J}}_1 = 	\begin{bmatrix}
	J_{1,\text{xx}} & 0 & 0\\
	0 & J_{1,\text{yy}} & 0\\
	0 & 0 & J_{1,\text{zz}}\\
	\end{bmatrix}= \begin{bmatrix}
	0 & 0 & 0\\
	0 & J_1 & 0\\
	0 & 0 & J_1\\
	\end{bmatrix}.
	\end{align}
\end{subequations}
Further simplifications are obtained by making the following substitutions. The total moment of inertia of the arm and the pendulum about the pivot point are
\begin{subequations}
		\begin{align}
			I_0 = J_0 + m_0l_0^2\\
			I_1 = J_1 + m_1l_1^2
		\end{align}
\end{subequations}
Evaluating the terms of Euler-Lagrange equations (\ref{ELequation}) with all substitutions above, gives the following equations of motion
\begin{subequations}\label{equations:motion}
	\begin{align}
	\ddot{\theta}_0 &= \frac{\gamma(\epsilon\dot{\theta}_0^2+\rho)-\delta(\tau+\beta\dot{\theta}_1^2-\sigma\dot{\theta}_0\dot{\theta}_1)}{\gamma^2-\alpha\delta}\label{motion1},\\
	\ddot{\theta}_1 &= \frac{\gamma(\tau+\beta\dot{\theta}_1^2-\sigma\dot{\theta}_0\dot{\theta}_1)-\alpha(\epsilon\dot{\theta}_0^2+\rho)}{\gamma^2-\alpha\delta}\label{motion2},
	\end{align}
\end{subequations}
where
\begin{subequations}
	\begin{align}
	\alpha &= I_0+L_0^2m_1+l_1^2m_1\sin^2\theta_1,\\
	\beta &= L_0m_1l_1\sin\theta_1, \\
	\gamma &= L_0m_1l_1\cos\theta_1,\\
	\delta &= I_1+l_1^2m_1,\\
	\epsilon &= l^2_1m_1\sin\theta_1\cos\theta_1,\\
	\rho &= m_1gl_1\sin\theta_1,\\
	\tau &= 2l^2_1m_1\sin\theta_1\cos\theta_1.
	\end{align}
\end{subequations}
Now, form equations of motian (\ref{equations:motion}),the non-linear dynamic model in a form of state-space model can be derived.
\subsubsection{State-Space Represntation} 
To obtain the state representation of the process the state variables must be defined first:
\begin{equation}
\begin{bmatrix}
x_1(t)&x_2(t)&x_3(t)&x_4(t)
\end{bmatrix}^\intercal = 
\begin{bmatrix}
\theta_0(t)&\dot{\theta}_0(t)&\theta_1(t)&\dot{\theta}_1(t)
\end{bmatrix}^\intercal.
\end{equation}
And the control variable:
\begin{equation} u(t) = \tau(t). \end{equation}

Equations (\ref{motion1}) and (\ref{motion2}) are further utilized to establish the state
space model
\begin{subequations}
	\begin{align}
	\dot{x}_1 &= \dot{\theta}_0, \\
	\dot{x}_2 &= \frac{\gamma(\epsilon\dot{\theta}_0^2+\rho)-\delta(\tau+\beta\dot{\theta}_1^2-\sigma\dot{\theta}_0\dot{\theta}_1)}{\gamma^2-\alpha\delta},\\
	\dot{x}_3 &= \dot{\theta}_1,\\
	\dot{x}_4 &= \frac{\gamma(\tau+\beta\dot{\theta}_1^2-\sigma\dot{\theta}_0\dot{\theta}_1)-\alpha(\epsilon\dot{\theta}_0^2+\rho)}{\gamma^2-\alpha\delta}.
	\end{align}
\end{subequations}
Now these non-linear differential equations could be written in the form of matrices:
\begin{equation}\label{nonlinmodel}
\begin{bmatrix}
\dot{x}_1(t) \\ \dot{x}_2(t) \\ \dot{x}_3(t) \\ \dot{x}_4(t)
\end{bmatrix} = \begin{bmatrix}
\dot{\theta}_0\\
\frac{\gamma(\epsilon\dot{\theta}_0^2+\rho)-\delta(\tau+\beta\dot{\theta}_1^2-\sigma\dot{\theta}_0\dot{\theta}_1)}{\gamma^2-\alpha\delta}\\
\dot{\theta}_1\\
\frac{\gamma(\tau+\beta\dot{\theta}_1^2-\sigma\dot{\theta}_0\dot{\theta}_1)-\alpha(\epsilon\dot{\theta}_0^2+\rho)}{\gamma^2-\alpha\delta}
\end{bmatrix}.
\end{equation}
Or, in the short form
\begin{equation} \dot{x}(t) = f(x(t),u(t)). \end{equation}
And this is the non-linear dynamic model of the Furuta pendulum, where the derivatives of the states are the functions of the current states and control input. And, also, there are nonlinear interactions between them. 
\subsection{Linear Dynamic Model}
Dynamics of the process could also be described in a simpler, but also less precise way~-~by the linear dynamic model. It can be obtained by linearising the non-linear model (\ref{nonlinmodel}) at some operation point. Then by a linear, time-invariant state-space model can be obtained in the following form
\begin{equation}\dot{x}(t) = \ui{A}{c}x(t) + \ui{B}{c}u(t).\end{equation}
Its constant matrices can be obtained by calculating its partial derivatives with respect to the each state and control variables.
\begin{subequations}\label{linSSmodel:general}
	\begin{align}
	\ui{A}{c} &= \begin{bmatrix}
	\frac{\partial f_1(x(t),u(t))}{\partial x_1}&\frac{\partial f_1(x(t),u(t))}{\partial x_2}&\frac{\partial f_1(x(t),u(t))}{\partial x_3}&\frac{\partial f_1(x(t),u(t))}{\partial x_4}\\
	\frac{\partial f_2(x(t),u(t))}{\partial x_1}&\frac{\partial f_2(x(t),u(t))}{\partial x_2}&\frac{\partial f_2(x(t),u(t))}{\partial x_3}&\frac{\partial f_2(x(t),u(t))}{\partial x_4}\\
	\frac{\partial f_3(x(t),u(t))}{\partial x_1}&\frac{\partial f_3(x(t),u(t))}{\partial x_2}&\frac{\partial f_3(x(t),u(t))}{\partial x_3}&\frac{\partial f_3(x(t),u(t))}{\partial x_4}\\
	\frac{\partial f_4(x(t),u(t))}{\partial x_1}&\frac{\partial f_4(x(t),u(t))}{\partial x_2}&\frac{\partial f_4(x(t),u(t))}{\partial x_3}&\frac{\partial f_4(x(t),u(t))}{\partial x_4}
	\end{bmatrix},\\ 
	\ui{B}{c} &= \begin{bmatrix}
	\frac{\partial f_1(x(t),u(t))}{\partial u}\\\frac{\partial f_2(x(t),u(t))}{\partial u}\\\frac{\partial f_3(x(t),u(t))}{\partial u}\\\frac{\partial f_4(x(t),u(t))}{\partial u}
	\end{bmatrix}.
	\end{align}
\end{subequations}
And when we compute these derivatives, we obtain a linearized continuous-time dynamic model of the pendulum in the form of state matrices
\begin{subequations}\label{linmatrices}
	\begin{align}
	\ui{A}{c} &=\begin{bmatrix}0&1&0&0\\
	0&0&\frac{-gL_0l_1^2m_1^2}{(m_1L_0^2+I_0)(m_1l_1^2+I_1)-L_0^2l_1^2m_1^2}&0\\
	0&0&0&1\\
	0&0&\frac{gl_1m_1(m_1L_0^2+I_0)}{(m_1L_0^2+I_0)(m_1l_1^2+I_1)-L_0^2l_1^2m_1^2}&0
	\end{bmatrix},\\
	\ui{B}{c} &=	\begin{bmatrix}
	0\\ 
	\frac{m_1L_1^2+I_1}{(m_1L_0^2+I_0)(m_1l_1^2+I_1)-L_0^2l_1^2m_1^2}\\
	0\\
	\frac{-L_0l_1m_1}{(m_1L_0^2+I_0)(m_1l_1^2+I_1)-L_0^2l_1^2m_1^2}
	\end{bmatrix},\\
	C &= \begin{bmatrix}0&0&1&0\end{bmatrix},\\
	D &= 0.
	\end{align}
\end{subequations}
At this point, as both linear and non-linear dynamic models of the pendulum are obtained, we are ready to proceed to the development of control strategies and controller design.
\section{Heuristic Swing-Up Control}\label{Hswing:teoria}
The Heuristic Swing-Up control strategy considers the control of the pendulum in tow steps. First is to bring the pendulum from the steady downside position to the vicinity of the upright position by an Energy-Shaping controller. Second is the stabilization of the pendulum by a predictive controller.
\subsection{Model Predictive Control}\label{mpcsection}
The Model Predictive Control (MPC) uses a model of the system to make predictions about the system’s future behavior. MPC solves an online optimization algorithm to find the optimal control action that drives the predicted output to the reference. The MPC also handles mulltiple-input and multiple-output (MIMO) systems that may have interactions between their inputs and outputs. It can also handle input and output constraints. The MPC has preview capability; it can incorporate future reference information into the control problem to improve controller performance. Due to all these properties MPC provides the highest quality of control performance at the moment.
\subsubsection{Model Predictive Control Formulation}
The model predictive control requires the linear discrete-time state-space model of the process
\begin{subequations}\label{linmodel}
	\begin{align}	
	x_{k+1} = Ax_k + Bu_k,\\
	y_k = Cx_k + Du_k.
	\end{align}
\end{subequations}
Thanks to the knowledge of that model, MPC predicts and optimize the evolution of states and outputs of the system over the prediction horizon.\\
Due to the setup of the controlled process, the MPC should be formulated to regulate the states of the system to the origin. And such MPC can be formulated as
\begin{subequations}\label{mpcgeneral}
	\begin{align}
		\min_{u_0,...,u_{N-1}}\ &\sum_{k=0}^{N-1} \lrp{ x_{k}^\intercal\ui{Q}{x}x_{k}+u_{k}^\intercal\ui{Q}{u}u_{k}}\\
	    \label{eq217b}\text{s.t.}\quad\,\ &x_{k+1} = Ax_{k} + Bu_{k}\qquad\quad  k \in \mathbb{N}_0^{N-1}\\
		&x_0 = x(0)\\
		\label{cst_x}&x_{k}\in\mathcal{X}\qquad\qquad\qquad\qquad\,  k \in \mathbb{N}_0^{N-1}\\
		\label{cst_u}&u_{k}\in\mathcal{U}\qquad\qquad\qquad\qquad\;   k \in \mathbb{N}_0^{N-1}
	\end{align}
\end{subequations}
where $N$ is a length of the prediction horizon, which defines the number of future control intervals the MPC must evaluate by prediction. The $N$ must be big enough to cover the closed-loop response of the process to prevent stability issues. Likewise, with the wider prediction horizon also increases the quality of control performance. On the other hand, the length of the prediction horizon greatly influences the solving time, so the compromise should be found. Matrices $\ui{Q}{x}$ and $\ui{Q}{u}$ are the weighting matrices, representing the penalty on the offset from the reference value, which in this case is equal zero and amount of energy, invested into the conrtol inputs. The $\mathcal{X}$ and $\mathcal{U}$ are polytopic state and input constraints respectively and defined as
\begin{subequations}
	\begin{align}
	\mathcal{X} &= \{\ui{H}{x}x_k\leq \ui{K}{x}\},\\
	\mathcal{U} &= \{\ui{H}{u}u_k\leq \ui{K}{u}\}.
	\end{align}
\end{subequations}
For case of box constraints
\begin{subequations}
	\begin{align}
		\ui{x}{min}\leq x_k \leq\ui{x}{max},\\
		\ui{u}{min}\leq x_k \leq\ui{u}{max},\\
	\end{align}
\end{subequations}
the matrices $\ui{H}{x}$, $\ui{K}{x}$, $\ui{H}{u}$ and $\ui{K}{u}$ have the following form
\begin{subequations}
	\begin{align}
		\ui{H}{x} = \begin{bmatrix}
						\ \ \,I\\-I
					\end{bmatrix},\quad & 
		\ui{K}{x} = \begin{bmatrix}
						\ \ \,\ui{x}{max}\\-\ui{x}{min}
					\end{bmatrix},\\
		\ui{H}{u} = \begin{bmatrix}
						\ \ \,I\\-I
					\end{bmatrix},\quad & 
		\ui{K}{u} = \begin{bmatrix}
						\ \ \,\ui{u}{max}\\-\ui{u}{min}
					\end{bmatrix}.
	\end{align}
\end{subequations}
Unfortunately, such formulation is inappropriate for quadratic programming solver, so it has to be reformulated in the form of a quadratic optimization problem. More detailed information could be find in~\cite{MPC:kniha}.
\subsubsection{Model Predictive Control as a Quadraic Programming Optimization Problem}
To solve an MPC problem, the Quadratic Programming (QP) solver is used. Unfortunately, an MPC formulation (\ref{mpcgeneral}) is inappropriate for QP solver and has to be reformulated. The standard QP problem has the following form
\begin{subequations}
	\begin{align}
	\min_{z}\quad & z^\intercal Pz + 2Q^\intercal z + R\\
	\label{cst_qp}\text{s.t.}\quad&Hz\leq G\\
	&\ui{H}{eq}z = \ui{G}{eq}
	\end{align}
\end{subequations}
As the first step, the standard MPC cost function should be written in a vector form
\begin{equation}
	\min_{U}\quad X^\intercal\ui{\tilde{Q}}{x}X + U^\intercal\ui{\tilde{Q}}{u}\,U\\
\end{equation}
where $X$ is a vector of predicted states, $U$ is an optimal trajectory of predicted control inputs and $\ui{\tilde{Q}}{x}$ and $\ui{\tilde{Q}}{u}$ are the matrices of original weight matrices.
\begin{equation}
	X = \begin{bmatrix}
	x_0\\x_{1}\\\vdots\\x_{N-1}
		\end{bmatrix},
\end{equation}
\begin{equation}
U = \begin{bmatrix}
u_0\\u_{1}\\\vdots\\u_{N-1}
\end{bmatrix},
\end{equation}
\begin{equation}
\ui{\tilde{Q}}{x} = \begin{bmatrix}
\ui{Q}{x}&0&\cdots&0\\
0&\ui{Q}{x}&\cdots&0\\
\vdots&\vdots&\ddots&\vdots\\
0&0&\cdots&\ui{Q}{x}
\end{bmatrix},
\end{equation}
\begin{equation}
\ui{\tilde{Q}}{u} = \begin{bmatrix}
\ui{Q}{u}&0&\cdots&0\\
0&\ui{Q}{u}&\cdots&0\\
\vdots&\vdots&\ddots&\vdots\\
0&0&\cdots&\ui{Q}{u}
\end{bmatrix}.
\end{equation}
At the next step, the equality constrains (\ref{eq217b}) should be expressed in the vector form. We can achieve that by predicting the evolution of states over the whole prediction horizon.
\begin{equation}
\begin{split}
x_0 &= x(t)\\
x_{1} &= Ax_0 + Bu_0\\
x_{2} &= Ax_{1} + Bu_{1}\\
&= A^2x_0 + ABu_0 + Bu_{1}\\
x_{3} &= Ax_{2} + Bu_{2}\\
&= A^3x_0 + A^2Bu_0 + ABu_{1} + Bu_{2}\\
&\vdots\\
x_{k} &= A^kx_0+\sum_{j=0}^{k-1}A^jBu_{k-j-1}
\end{split}
\end{equation}
Next, these equations could be written in a compact form
\begin{equation}
	\begin{bmatrix}
	x_0\\x_{1}\\ x_{2}\\\vdots\\ x_{N-1}
	\end{bmatrix} = 
	\begin{bmatrix}I\\A\\A^2\\ \vdots \\ A^{N-1}\end{bmatrix}x_0 + 
	\begin{bmatrix}
	0& 0&\cdots&0&0\\
	B&0&\cdots&0&0\\
	AB&B&\cdots&0&0\\
	\vdots&\vdots&\ddots&\vdots&\vdots\\
	A^{N-2}B&A^{N-3}B&\cdots&B&0\end{bmatrix}
	\begin{bmatrix}u_0\\u_{1}\\u_{2}\\\vdots\\u_{N-1}\end{bmatrix}.
\end{equation}
Or in the short form
\begin{equation}\label{eq227}
	X = \tilde{A}x_0 + \tilde{B}U.
\end{equation}
At this point, we can substitute as $X$ in the cost function by (\ref{eq227}). Then we obtain the new objective function
\begin{equation}\label{eq228}
	\min_{U}\ (\tilde{A}x_0 + \tilde{B}U)^\intercal\ui{\tilde{Q}}{x}(\tilde{A}x_0 + \tilde{B}U) + U^\intercal\ui{\tilde{Q}}{u}\,U.
\end{equation}
And when we expand and simplfie (\ref{eq228}), we obtain
\begin{equation}
	U^\intercal(\tilde{B}^\intercal\ui{\tilde{Q}}{x}\tilde{B} + \ui{\tilde{Q}}{u})U + 2x_0^\intercal\tilde{A}^\intercal\ui{\tilde{Q}}{x}\tilde{B}U + x_0^\intercal\tilde{A}\ui{\tilde{Q}}{x}\tilde{A}x_0.
\end{equation}
Now if we define $z = U$, we can cleary see matrices $P$, $Q$ and $R$, which occurs in a standart cost function for the quadratic optimization.
\begin{subequations}
	\begin{align}
		P &= \tilde{B}^\intercal\ui{\tilde{Q}}{x}\tilde{B} + \ui{\tilde{Q}}{u},\\
		Q &= (2x_0^\intercal\tilde{A}^\intercal\ui{\tilde{Q}}{x}\tilde{B}U)^\intercal,\\
		R &= x_0^\intercal\tilde{A}\ui{\tilde{Q}}{x}\tilde{A}x_0.
	\end{align}
\end{subequations}
Now only constrains remain. Constrains (\ref{cst_x}) and (\ref{cst_u}) must be reformulated as (\ref{cst_qp}). Those constrains we consider as a upper and lower bounds for the states and control inputs respectively
\begin{subequations}
	\begin{align}
		\ui{x}{min} &\leq x_k \leq \ui{x}{max} \quad k \in \mathbb{N}_0^{N-1}\\
		\ui{u}{min} &\leq u_k \leq \ui{u}{max} \quad k \in \mathbb{N}_0^{N-1}
	\end{align}
\end{subequations}
Now we split and vectorise those constrains
\begin{subequations}
\begin{align}
	X &\leq \ \; \, \ui{X}{max},\\
	-X &\leq -\ui{X}{min},\\
	U &\leq \ \; \, \ui{U}{max},\\
	-U &\leq -\ui{U}{min}.
\end{align}
\end{subequations}
In the next step, $X$ in the states constrains is substituted by a (\ref{eq227}) and $U$ expressed
\begin{subequations}
	\begin{align}
	\tilde{B}\,U &\leq \ \; \,\ui{X}{max} - \tilde{A}x_0,\\
	-\tilde{B}\,U &\leq -\ui{X}{min} + \tilde{A}x_0,\\
	U &\leq \ \; \,\ui{U}{max},\\
	-U &\leq -\ui{U}{min}.
	\end{align}
\end{subequations}
Those constraints are in the standard form and by combining them, we obtain matrices of inequality constraints $H$ and $G$ for the quadratic programming problem
\begin{equation}
	H = \begin{bmatrix}
	\ \; \,\tilde{B}\\
	-\tilde{B}\\
	\ \ \, I\\
	-I\\
	\end{bmatrix}, \quad
	G = \begin{bmatrix}
	\ \; \,\ui{X}{max} - \tilde{A}x_0\\
	-\ui{X}{min} + \tilde{A}x_0\\
	\ \ \:\ui{U}{max}\\
	-\ui{U}{min}
	\end{bmatrix}.
\end{equation}
At this point, as matrices $R$, $Q$, $R$, $H$ and $G$ are defined, and MPC strategy can be implemented via using any QP solver.
\subsection{Energy Shaping Controller}\label{energyshapingsection}
For the initial excitation of the system, we use the energy-based swing-up controller~\cite{furuta:swing}. The strategy with this controller is that we increase the amplitude of swings by increasing the energy of the system with every swing. The energy is added by controlling arms movements and depends on the actual energy of the pendulum. The actual energy of the pendulum can be calculated from the actual position of the pendulum and its velocity: 
\begin{equation}
E = \frac{m_1gl_1}{2}\lrp{\lrp{\frac{\dot{\theta}_1}{\omega_0}}^2+\cos\theta_1 - 1}.
\end{equation}
Than the control law has following form:
\begin{equation}\label{energy-shaping}
	u = \ui{k}{v}E\,\mathrm{sign}\lrp{\dot{\theta}_1\cos\theta_1}.
\end{equation}
Where element $\mathrm{sign}\lrp{\dot{\theta}_1\cos\theta_1}$ determines direction i which the force will be applied and $\ui{k}{v}$ is the gain of the controller.
\section{Optimal Swing-Up Control}\label{nmpcsection}
Unlike the Heuristic Swing-Up control strategy in Optimal Swing-Up control strategy only one controller - a Non-linear Model Predictive Controller (NMPC) is used to control the pendulum.\\

Generally speaking, any MPC uses a model of the process to predict the evolution of states and outputs of the process in the following sampling periods. And in general, the dynamics of the real process are complex and cannot be described perfectly. Any modeling method introduces some uncertainties into the resulting model. For example, the linear model is valid only in som range around the linearization point. So linear MPC, which uses such could be designed to control the process only in that range. But nonlinear modeling is a much more precise approach to describe the process dynamic. And in the case of Furuta pendulum, its full range dynamic can be described by a nonlinear model (\ref{nonlinmodel}). And by using such a model, an NMPC strategy is capable to perform Swing-Up control of the pendulum by itself.\\
Any predictive controller solves an optimization problem to ensure the best control performance. In case of NMPC, such optimization problem is a Nonlinear Programing (NLP) problem. Usually, such optimization problems are extremely difficult to solve analytically. And in this section a numerical method to solve a NLP problem is described. To illustrate the method, the general form of NLP problem, with the $x$ as an arbitraty optimization variable, is considered
\begin{subequations}\label{nlpgeneral}
	\begin{align}
	\min_{x}\  &f(x)\\
	\text{s.t.}\  &h(x) = 0\\
		 &g(x)\leq 0
	\end{align}
\end{subequations}
In this thesis a numerical method called Sequential Quatratic Programming is used to solve an NMPC problem. 
\subsubsection{Sequential Quatratic Programming}\label{SQP:theory}
Sequential Quadratic Programming (SQP) is arguably the most successful method for solving nonlinearly constrained optimization problems~\cite{SQP:Theory}. Its general idea is to approximate NLP at current iterate by QP subproblem, solve that subproblem, than use the solution to construct a new iterate. This construction is done in such a way that the sequence converges to a local minimum of the NLP.

The main condition for a QP subproblem is that it has to reflect the properties of NLP at current iteration. But before such subproblem could be constructed, a few terms should be establish first.
\begin{itemize}
	\item Lagrangian functional of the NLP, $\mathcal{L}$ is defined as\begin{equation}
		\mathcal{L}(x,\lambda,\mu) = f(x) + \lambda^\intercal h(x) + \mu^\intercal g(x). 
	\end{equation}
	\item The gradient of $f$, $\nabla f(x)$ is denoted as
	\begin{equation}
		\nabla f(x) = \lrp{\diff{f(x)}{x_1},\diff{f(x)}{x_2},\cdots,\diff{f(x)}{x_n}}^\intercal.
	\end{equation}
	\item Hessian of f, $\nabla^2f(x)$ is the matrix of second partial derivatives as given by
	\begin{equation}
		\lrp{\nabla^2f(x)}_{ij} = \diffxy{f(x)}{x_i}{x_j}.
	\end{equation}
\end{itemize}
A major concern in SQP method is the choice of appropriate quadratic subproblem. 
The most obvious choice for the objective functional in this quadratic program is its local quadratic approximation
\begin{equation}
f(x)\approx f(x_k)+\nabla f(x_k)\lrp{x-x_k}+\frac{1}{2}\lrp{x-x_k}^\intercal \nabla^2f(x_k)\lrp{x-x_k}.
\end{equation}
and the constraint functions $h$ and $g$ by their local affine approximations
\begin{subequations}
	\begin{align}
		h(x)&\approx h(x_k)+\nabla h(x_k)\lrp{x-x_k},\\
		g(x)&\approx g(x_k)+\nabla g(x_k)\lrp{x-x_k}.	
	\end{align}
\end{subequations}
And after setting $d_x$ as a solutiion of QP problem and matrix $B_k$ as the hessian matrix in each iteration
\begin{subequations}
	\begin{align}
		d_x&=x-x_k,\\
		B_k &= \nabla^2f(x_k),
	\end{align}
\end{subequations}
the following form of the QP subproblem is accured
\begin{subequations}
	\begin{align}
		\min_{d_x}\  &\nabla f(x_k)^\intercal d_x+\frac{1}{2}d_x^\intercal B_k d_x\\
		\text{s.t.}\  &h(x_k)+\nabla h(x_k)\lrp{x-x_k} = 0\\
		&g(x_k)+\nabla g(x_k)\lrp{x-x_k}\leq 0
	\end{align}
\end{subequations}
Such formulation is reasonable in case of linear constraints. In case of nonlinear constraints such choice is inappropriated. 

To take nonlinearities in the constraints into account a quadratic model of the Lagrangian function could be used as the obective instead.
\begin{subequations}
	\begin{align}
	\min_{x}\  & \mathcal{L}(x,\lambda^*,\mu^*)\\
	\text{s.t.}\  &h(x)= 0\\
	&g(x)\leq 0
	\end{align}
\end{subequations}
Although the optimal multipliers are unknown, then approximations $\lambda_k$,$\mu_k$ could be contained as part of the iterative process. Then at a current iterate $(x_k,\lambda_k,\mu_k)$, the objective functional in stems from the quadratic Taylor series approximation
\begin{equation}
	\mathcal{L}(x_k,\lambda_k,\mu_k)+\nabla\mathcal{L}(x_k,\lambda_k,\mu_k)^\intercal d(x)+\frac{1}{2}d_x^\intercal \nabla^2\mathcal{L}(x_k,\lambda_k,\mu_k)d_x.
\end{equation}
What leads to the final QP subproblem
\begin{subequations}
	\begin{align}
	\min_{d_x}\  &\nabla \mathcal{L}(x_k,\lambda_k,\mu_k)^\intercal d_x+\frac{1}{2}d_x^\intercal nabla^2\mathcal{L}(x_k,\lambda_k,\mu_k)d_x\\
	\text{s.t.}\  &h(x_k)+\nabla h(x_k)\lrp{x-x_k} = 0\\
	&g(x_k)+\nabla g(x_k)\lrp{x-x_k}\leq 0
	\end{align}
\end{subequations}
By solving that QP subproblem solution $d_x$ as well as local optimal values of multipliers $\ui{\lambda}{qp}$ and $\ui{\mu}{qp}$ are accuried, and setting 
\begin{subequations}
	\begin{align}
		d_\lambda &= \ui{\lambda}{qp}-\lambda_k,\\
		d_\mu &= \ui{\mu}{qp}-\mu_k.
	\end{align}
\end{subequations}
allow the updates of $(x,\lambda,\mu)$ to be written in the compact form
\begin{subequations}
	\begin{align}
	x_{k+1} &= x_k + \alpha d_x,\\
	\lambda_{k+1} &= \lambda_k + \alpha d_\lambda,\\
	\mu_{k+1} &= \mu_k + \alpha d_\mu.
	\end{align}
\end{subequations}
for some selection of the steplength parameter $\alpha$. Steplength $\alpha$ should be chosen that the following equation is fullfiled
\begin{equation}\label{steplength}
	m(x_k + \alpha d_x)\leq m(x_k).
\end{equation}
Where $m$ is a merit function, which is a scalar-valued function of $x$ that indicates whether a new iterate is better or worse than the current one. Also, an appropriate steplength $\alpha$ that decreases $m$ can be computed, for example by a ``backtracking'' procedure of trying successively smaller values of $\alpha$ until a suitable one is obtained. In unconstrained minimization, there is a natural merit function namely the objective function itself. In the constrained setting, unless the iterates are always feasible, a merit function has to balance the drive to decrease $f$ with the need to satisfy the constraints. This balance is often controlled by a parameter in that weights a measure of the infeasibility against the value of either the objective function or the Lagrangian function. The merit function used in this thesis is
\begin{equation}
m(x,p) = f(x) + p\left\|e(x)\right\|_1.
\end{equation}
Where $f(x)$ is the objective function of (\ref{nlpgeneral}), $e(x)$ contains all constraints in (\ref{nlpgeneral}) and $p$ is a penalty parameter.

Lastly, the global convergence proofs for the resulting SQP algorithm are generally similar to those found in unconstrained optimization. For example 
\begin{equation}
	|m(x_{k+1},p) - m(x_{k},p)|\leq 1\mathrm{e}{-6},
\end{equation}
or
\begin{equation}
	\lim_{k\to\infty} \nabla m(x_{k},p) = 0
\end{equation}
This implies that $\{x_k\}$ converges to a critical point of $m$.\\

In summary the basic SQP algorithm looks like this:

\begin{algorithm}[H]
	\caption{Basic SQP algorithm}\label{SQP:algorithm}
	\begin{algorithmic}[1]
	\Procedure{SQP}{$x_0,\lambda_0,\mu_0,B_0$}
	\State $k\gets 0$
	\While{not converged}
	\State Form and solve QP subproblem to obtain $(d_x,d_\lambda,d_\mu)$.
	\State Choose steplength $\alpha$ so that equation (\ref{steplength}) is satisfied
	\State 	$x_{k+1} \gets x_k + \alpha d_x$,
	\State $\lambda_{k+1} \gets \lambda_k + \alpha d_\lambda$,
	\State $\mu_{k+1} \gets \mu_k + \alpha d_\mu$.
	\State Compute $B_{k+1}$.
	\State $k \gets k+1$.
	\EndWhile
	\EndProcedure	
	\end{algorithmic}
\end{algorithm}	
And with the SQP theory explained, all theoretical foundations are made. And we can proceed to the next main part of the thesis, where all the theories above will be used to develop control strategies and finally control the Furuta pendulum.